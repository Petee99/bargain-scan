The technology of scanning barcodes has evolved since inventing it. Nowadays, these codes have multiple types, such as the one-dimensional, linear barcodes, the two dimensional barcodes, like QR code and Data Matrix \cite{}. These codes are scannable with either a special scanning device, a mobile phone or some other imaging devices, which have a software capable of deciphering these codes installed.

While websites with capabilities to track product details do it so using a code from user input, the advent of smart phones with built-in cameras has led to designing applications which are capable of fast scanning of barcodes and then fetching details about the given item \cite{}. Throughout the years many libraries and APIs have been developed to make integrating such functions into mobile applications easier.

No matter whether the web or the mobile approach is used, these applications could provide crucial functionalities when it comes to the rising trend of barcode scanning and real time price comparison. Soon after realizing this, many companies decided to provide users with solutions with such functionalities. However, to come up with something innovative, we have to understand the background and functionality of already existing software. This chapter will give a comprehensive analysis on multiple similar or functionally related solutions, such as PriceGrabber.com, Google Shopping, RedLaser, PriceRunner, ShopSavvy, and Hungarian ones like Arukereso.hu and Cashmap.hu. Considering these implementations, my goal is to develop an application that will stand out in the competitive market.

\subsection{PriceGrabber.com}

Established in 1999, PriceGrabber is one of the oldest price comparison services out there. Naturally, back then it started off as a website, but later a mobile application was developed as well. The service lets customers get free price information on millions of products, while getting its profit from countless merchants paying for each lead. In 2005 it was acquired for 485 million USD by Experian and was expected to grow substantially in the next five years \cite{Forbes2005}, then later another acquisition was made by Connexity in 2015 and it has been owned by them ever since \cite{Connexity2015}.

The success of PriceGrabber originates from its ability to adapt through the evolution of the technology industry. Eventually a smart phone application was released, first for iPhones, later for android devices, which enabled the user to access the same functions as from the website. It was designed to host an increasing number of users and allowed them to read product reviews, create a favorite list and compare prices. Laura Conrad, the president of PriceGrabber, stated that this application was developed to remove the boundaries between online and offline shopping, targeting the growing number of people of mobile shoppers \cite{HarnickPriceGrabberApp}.

\subsection{Google Shopping}

Possibly the best known e-commerce platform, Google Shopping took off in 2002. It has been a major part of the search engine and google ads ever since. Whenever a user searches for something retail related, google gives them the best matches and options alongside with sponsored ads.

On the business side, the target audience is made up mainly from advertisers and sellers, providing them with different methods to promote their goods and services. The service is based on a Cost-Per-Click model, which means that Google gets paid for every click that users make on a company's ad. Ranking is affected by multiple factors, such as relevance to search keywords, how well does it fit into the user search history and last, but not least on how much does a company pay for their product to be ranked higher, in which case google differentiates it with a "Sponsored" flag from the other ads.

From the consumer's point of view, the service provides personalized ads based on factors like demography, search history and interactions with Google services. This kind of personalization ensures relevance and user engagement leading to a competitive, yet rewarding environment.

On the technical note, the service is backed up by Google's robust systems. It is strict with forcing the retailers adhere to Google's policies before letting them post their ads. It uses machine learning models to maximize the relevance for users. Google Shopping also gives a lot of personalization possibilities for the consumers, such as filtering the products, depending on the category, brand and more importantly price. The service also incorporates reviews from multiple sources to assist a better understanding on the benefits and drawbacks of choosing certain products. The process ends with clicking an ad, where Google's business model steps into action and the seller pays them for the successful redirection to their product's page. \cite{Google2023}

\subsection{RedLaser}
Although it has been discontinued since and taken down from both App Store and Google Play Store, RedLaser, acquired by eBay in 2010 was one of the most popular barcode scanner and price comparison application using various services of its acquirer.

In 2011 RedLaser 3.0 rolled out, with integration of various eBay mobile services, creating a great user experience for shopping based on price comparison. The application utilized Milo's local inventory data to enable price comparison among online and offline stores and PayPal's mobile express checkout for a secure payments.

Although the earlier versions were well known too, the application's popularity rapidly grew thanks to the new, unique features introduced by this update. These included the ones mentioned in the previous paragraph, along with a refined user interface, which made scanning and creating lists and QR codes much easier. It also enabled users to share categorized shopping carts through Facebook, SMS and e-mail.

In RedLaser's prime it was downloaded tens of millions of times, however, after many years, it could not keep up with the competition and was discontinued.

\subsection{PriceRunner}

PriceRunner, founded in 1999, is an independent price comparison service offering a platform to help consumers make informed decisions about their purchases in the United Kingdom. Using the service is free of charge and it provides daily updates on millions of products regarding their price and details. Even though the in 2022 PriceRunner was merged into Klarna, PriceRunner managed to keep its core values, which are independence, credibility and unbiasedness, therefore highly contributing to smarter e-commerce decisions. \cite{PriceRunner2023}

Much like Google Shopping, PriceRunner uses the model of affiliate marketing \cite{PriceRunnerTerms2023}. This is commonly known as the Cost-Per-Click model mentioned earlier in Google's case. The service gets paid by affiliated customers after each successful referral. It boasts a unique feature, buyer protection, beneficial for the users in case where the retailer does not meet their statutory obligations. Buyer protection is a free service for all users, providing financial coverage up to five thousand pounds, in case of a purchase from any of PriceRunner's associated retailers. \cite{PriceRunnerFaq2023}

PriceRunner offers their services both on their website and on their mobile application. Being UK's largest product and price comparison service, it lines up more than 2.6 million products. The app provides the users with smart features enhancing their shopping experience. These functions include but are not limited to using its built-in barcode scanner to find certain items, compare their prices and upcoming deals on the product. It also equips the customer with price alert settings, so that they will be notified about the drops in price. The great user experience is maintained by continuously updating pricing data from thousands of retailers ensuring that the price comparisons are up-to-date. Additionally, users can create lists of certain products they like and monitor their prices and deals on them. \cite{PriceRunnerApp2023}

\subsection{ShopSavvy}

Possibly the best application out there in this topic would be ShopSavvy, developed by Monolith Technologies Inc., which has been around since 2008. It was one of the first applications developed for both iOS and Android. It is available for all popular browsers, such as Microsoft Edge, Google Chrome and Safari as an extension plus as a mobile application for iPhones, iPads and all sorts of Android devices. The service's goal is to provide the user with the best possible prices on all kinds of items, with promoting fairness, being customer oriented and independent. \cite{ShopSavvyAbout}

ShopSavvy evolved from a barcode scanner app to a service that rounds up tens of thousands of retailer partners and covers all aspects of price comparison such as sales, markdowns and coupons to provide the best possible price for the user. It monetizes the platform in two major ways. Firstly, they target users with personalized offers through the service and if this offer catches the eye of the consumer enough for them to make a purchase, the company will earn commission from that sale. Secondly, it generates revenue from advertisers using the platform to reach even more costumers, therefore serving as a mediator platform between retailers and customers. ShopSavvy considers itself as a product search company, with an aim to match the consumers with their desired products for the best possible price. CEO John Boyd explains that the best strategy for competing with giants like Google, Facebook and Amazon is to constantly innovate and come up with unique value propositions. \cite{PymntsShopSavvy}

The company rounds up multiple software solutions for using their service. Meanwhile the browser add-ons are for online shopping, usually for non-essential items, the application shines when it comes to grocery shopping with barcode scanning possibilities. The service itself gathers information from over 30 thousand retailers, both online and physical stores through two methods. Firstly, it partners up with retailers, so they would provide information about prices and sales, secondly it uses web crawling and scraping to find information on products from non-partner stores. The data about items and prices are collected real time, meaning that the user is provided with the most up-to-date information. The search and scraping work using both barcodes and keywords to find accurate results. The hardships of the recent lockdowns forced ShopSavvy to adapt and come up with an in-stock inventory tracking, which lets customers locate hard-to-find items in stock. It offers an easy to navigate platform for users to discover, shop and save. It aggregates product information from a wide-variety of network sources to provide the customer with unbiased data on deals, ratings and reviews. The platform also provides a place for the users to form a mobile shopping community that extends beyond the general concept of such applications. \cite{shopsavvy2020}

\subsection{Arukereso.hu}

Arukereso.hu is a market-leading price comparison website in Hungary with operating in two additional countries, namely Romania and Bulgaria. It started off as a small service maintained by two individuals in 2004, listing and comparing a few items from a few webshops around the dawn of internet shopping. In 2009 it was acquired by Allegro Group, but the current state was established in 2016 when another acquisition was made by Rockaway group. \cite{arukeresoLinkedin}

Arukereso.hu is similar to previous price comparison websites. It uses the same Cost-Per-Click affiliate business model to get revenue from partner retailers or sometimes extra for actual purchases through them. What makes them unique on the market is the fact that it is the best-known website with such functionality in Hungary, where many other companies do not operate yet, coupled with providing an all-around unbiased review system, where users can rate both the product and the seller. This system solves the asymmetry when it comes to cheap prices and questionable sellers and also ensures a more trustworthy line-up of stores. \cite{karacsony2019}

The previously mentioned review system comes into play after a user makes a purchase. They get asked to provide feedback on the product they purchased along with the seller they bought it from. The ratings are designed to allow replies, that way the stores or other users can clarify reviews in certain cases. The site also provides an option to ask questions about certain products, which paired with the review system provides a platform for a community of mindful customers. An additional layer of trust is provided by a "Trustworthy Shop" badge, which is only achievable with a high number of positive reviews left by actual users \cite{arukereso}. 

\subsection{Cashmap.hu}

With unfortunately less public information available comes another Hungary specific shopping assistant, Cashmap.hu. It is a newly launched startup designed to transform the way Hungarians shop. It encourages their users two make well informed and rational decisions about where they shop and what they buy. It provides information on prices and discounts for up-to 9 popular supermarket chains. The service is completely independent, it is not by any means in business relationship with the stores they provide information about. \cite{maradokapenzemnel2021}

Their service is a web-based application with features like searching for the desired product and compare prices among multiple stores. The user can also put a personalized shopping cart together and calculate which store offers the best final sum. If the shopping cart contains items the user does not need right away, there is a possibility to move them into a "Later Cart" to wait for future discounts. The service uses web crawling and scraping, using multiple sources, including in-store surveys, online sale fliers, online store interfaces and internet searches to ensure that the information provided is accurate and up-to-date. \cite{maradokapenzemnel2021}

\subsection{Conclusion}

After a thorough examination of already existing software solutions for this specific problem, it is clear, that the market is somewhat saturated and that there are multiple notably great features implemented in the analyzed applications. However, it is important to mention, that while some have great functions, there are those that do not exist anymore and others, which do not necessarily target the same demographics. 

Since the idea for creating this application came from the current situation in Hungary, specifically with grocery shopping in mind, while the international and other region specific implementations are important from a technical point of view, they do not affect the market and target audience in question yet. On the other hand, both of the Hungarian websites mentioned provide features that are on par with some of the ones I am planning to implement. This also encourages a solution for this problem that places the application in a unique spot on the market. A key difference will be accessibility, since the software will be a dedicated smart phone application instead of a generic website solution. 

Finally, the following conclusion can be drawn from this analysis. For the upcoming application to be successful among the target audience and in the target market segment, it needs to combine the best of all worlds, meaning that it should line-up a number of features not yet available in our region, build on the existing and well-established model, pay close attention to the reactions and requests of the user-base and adapt if needed. In the upcoming chapter I will carefully develop the vision for this application and its aspects.