The purpose of this thesis work is to explore the topic of creating a mobile application that would make people's lives easier when it comes to shopping. With the enormous growth of the competitiveness of the retail market, as well as e-commerce, customers are often left confused whether they would be better off just shopping in the nearest supermarket or rather travel to another one that is further away, but may offer better prices. Implementing a solution for this problem has the potential to become valuable in the eyes of confused customers, who would like to make informed decisions regarding their shopping habits.

Elaborating on the previously mentioned issue, the consumers often face multiple challenges when it comes to a seemingly easy task like shopping. Without information on prices, decision-making is only affected by factors like travel time and personal preference. For this reason, shopping can lead to overpaying for items or not finding the preferred brand or good in the selected store. The process can also become more time-consuming than expected, since there is no guarantee that the customer will only have to visit one shop. Big differences between retail prices can also raise trust issues and misconceptions among customers. Additionally, from the retailers perspective, coming up with the right pricing strategy can be hard, when the factors affecting the customers' decisions are unclear.

The development of such application can potentially result in numerous benefits for both the customer and retail side, such as clear comparisons between different retailers, a more transparent way of pricing the goods, efficient adaptation to the newest valuation trends, customer oriented marketing and the possibility to reduce the time spent with shopping. All of these would result in customers who are aware about the stocks and prices of these retail shops, leading to them making well-informed decisions when choosing shopping habits, as well as getting better prices than ever, because of the fiercer contest on providing the best prices.

The idea of making the life of consumers easier has been around for a while now, but many times they are web-based, meaning that the user experience is not going to be as optimized and straightforward as a dedicated application's case. Furthermore, these applications are usually targeted for international customers, or for those of other nationalities, not to mention that websites usually lack the possibility of searching for bar codes, let alone scanning them for an easy and fast experience.

Implementing the idea for this application requires coming up with all the possible use cases, a well planned  software architecture, a seamless development process, and the thorough testing of the end product. 

Within the scope of this thesis work project, an application will be designed which will provide the user with functions such as using the camera for scanning the bar code of an item, searching for these items with this code or product name, getting prices from various retailers. Moreover, the application will try to give personalized recommendations for the user on where to shop, depending on the goods availability, prices and distance of the shops.

While the primary aim of this project is to target Android devices, for future possibilities it might be more desirable to develop the application using a platform independent approach, like Xamarin. For achieving the desired functionality, there is a need for numerous APIs to provide relevant data on product information, prices as well as the distance between the user and surrounding shops. The testing of the product will be done through multiple approaches to ensure a seamless user experience.

In conclusion, this easy and accessible way of price comparison between shops, brands and goods can lead to an improved consumer culture, where both retailers and customers can make informed decisions. In the following chapters, a detailed examination will be provided on the topic of the importance of transparent price comparison, relevant fields and existing solutions, as well as outlining the concept of the project application, planning and executing the implementation, and finally testing the end product.



