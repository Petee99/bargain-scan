The development of barcode technology, as well as the upbringing of handheld software solutions already have noticeable effects on how we shop in our favorite stores. Many retail shops have already implemented self checkout services and handheld or stationary devices, which provide information on the scanned item. Taking this technology one step further will lead us to mobile applications. Out of many already existing programs, finding a Hungary-specific mobile application with such functionality is rather challenging.

\subsection{History of Barcode Technology}
The invention of barcode dates back to the 1940s, when Bernard Silver and Norman Woodland came up with the patent for a system that automatically read product information, using ultraviolet light and ink patterns \cite{barcode2023}. However, only after the introduction of Universal Product Code (UPC) in 1974 did it become widespread in retail industry \cite{weightman2015}. Since then, multiple new types of barcodes emerged, such as the Quick Response (QR) code, which is able to store substantially more information and is faster to read than the original version \cite{densowave2011}. 

\subsection{Evolution of Mobile Devices and Barcode Scanning Applications}

Since the mobile devices come with high-end cameras and very high computational capabilities, they led to the possibility of developing applications, that are capable of scanning the barcodes on products and then providing them with useful information about them, like details, ratings and prices \cite{chen2012}. The first famous pioneers were RedLaser and ShopSavvy, whose popularity has been mainly connected to the users ability to save time and money using them for comparing product details and prices \cite{rao2011}.

\subsection{Relevance of Price Comparison in The Retail Industry}

A highly competitive market results in price comparison becoming an essential aspect of modern shopping experience, since an increasing majority of consumers strive to become more informed about the products and their details before making a decision on where and what to shop for \cite{huang2006}. This trend is fueled by e-commerce, widespread mobile internet access and an increasingly high competitiveness in retail market \cite{pentina2014}. Using these applications led to retailers being forced to use different pricing strategies, as well as investing in digital technology to provide the customers with the required transparency and comfort \cite{grewal2011}.