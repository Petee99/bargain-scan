\subsection{Introduction}

\subsubsection{Purpose}
This chapter serves as an outline for the vision for "BargainScan", an innovative mobile application to help customers save on everyday shopping tasks and to spend on what really matters. This vision will guide the reader through all key aspects of the development of the upcoming application such as opportunities, challenges, target audience, benefits and features of the software. It will act as a reference for all stakeholders involved in the project.

\subsubsection{Scope}
"BargainScan" is an upcoming mobile application aimed to make grocery shopping a conscious and informed act for Hungarians by providing a platform for scanning and searching for products, comparing the prices across supermarket chains and retail shops, providing calculations on shopping lists in different stores, listing shops sorted by travel distance and affordability, and building an aware shopping community in Hungary.

\subsection{Positioning}

\subsubsection{Business Opportunity}
The global and local e-commerce and retail market has shown significant growth in the past decade. With the increasing adoption of smart phones into people's everyday life, especially concerning their shopping habits, it is clear that there is an opportunity for developing applications to assist them with these daily tasks. "BargainScan" is aimed to fulfill this mission by offering a comprehensive and versatile solution, equipped with a user-friendly and straightforward interface.

\subsubsection{Problem Statement} 
As stated in the previous chapter (\ref{related}), while the global market has many solutions for shopping assistants, many lack important features or are implemented as websites which are less accessible than a mobile application. Additionally, even if there are adequate solutions on the global market, they simply are not meant to be used in Hungary, since they only list products and retailers that are commonly not available in Hungary, with many times not even supporting this region.

Currently there are no mobile applications with such purpose on the Hungarian market. On the other hand, there is a series of already established mobile applications, that more or less offer useful features for the users, which share the problem of not being accessible enough, especially for older folks, who need a very straightforward way of accessing the software. Moreover, neither of the existing applications seem to offer a solution that covers the whole process of finding the best products and prices and planning the next shopping spree carefully including various factors, for example travel distance, or how crowded the place is.

\textbf{Customer Perspective:}
They need a comprehensive and intuitive application that helps them find not only the best prices, but also the best way of shopping by filtering out stores that do not fit into their desired characteristics. Additionally, they need a platform where they can build a community to rate and share insights on certain products and sellers.

\textbf{Seller Perspective:}
Businesses need a platform to provide their up-to-date information on, therefore reaching potential customers, increasing the likeliness of being chosen by the users and making their prices competitive against other companies.

\subsubsection{Product Position Statement}

"BargainScan" will be an application designed for money-conscious individuals who seek a comprehensive, clean and user-friendly solution for the stated problem. Unlike existing applications, its primary aim is to cover not only a few, but most aspects of the shopping process, from searching for products through their name or barcode, finding the best prices, filtering shops and products on certain terms, getting personalized recommendations and reading unbiased ratings and opinions on products and businesses. 

While realizing all of the features that will make this software a unique solution, it is important to provide an application with a straightforward and easy-to-use interface for the target audience, because accessibility plays a key aspect in the adaption process, hence in the general success.

\pagebreak

\subsection{Market Analysis}

\subsubsection{Market Demographics}

The growth in the competitiveness of retail and e-commerce led to a situation where customers are overwhelmed by choices, but they are unable to make informed ones due to lack of information about stores and products. The application is aimed to target the individuals with the desire to make conscious decision about their shopping habits. In the table (\ref{tab:md}) below, I will list the key aspects of the target market demographic.

\begin{table}[h]
	\centering
	\begin{tabularx}{\textwidth}{|p{3cm}|X|}
		\hline
		\textbf{Aspect} & \textbf{Description} \\
		\hline
		Age Group & The target age group is between 12 and 99, comprising Generation Z, millennials, Generation X and even baby boomers, meaning that the application is for everyone that seeks help with making informed decisions. Additionally, the intuitive design of the application makes it suitable and accessible for all age groups. \\
		\hline
		Income Level & The application can prove to be useful for customers across various income levels, offering features for comparing prices and finding the best deals, which is particularly appealing to middle-income earners. \\
		\hline
		Geographic Location & The shopping assistant app is primarily targeted towards all areas of Hungary with internet connection and grocery stores available. \\
		\hline
		Shopping Habits & The primary user base is expected to be mostly made up from millenials and Gen X members, since they are shopping for groceries on a daily basis, feel more comfortable using technological solutions in their everyday routines and value convenience and time saved. \\
		\hline
		Technological Proficiency & The application will be developed for smartphone users with different levels of technological proficiency. The user-friendly interface is set out to ensure that every individual will be able to wrap their head around the concepts of the application.\\
		\hline
	\end{tabularx}
	\caption{Market Demographics}
	\label{tab:md}
\end{table}

\pagebreak

\subsubsection{Porter's Five Forces Analysis}

To better understand the competitive dynamics and attractiveness of the proposed "BargainScan" application on the market, Porter's Five Forces analysis will be utilized in the table(\ref{tab:pffa}) below, encompassing the threat of new entrants, bargaining power of suppliers, bargaining power of buyers, threat of substitute products or services and rivalry among existing competitors.

\begin{table}[h]
	\centering
	\begin{tabularx}{\textwidth}{|p{4cm}|X|}
		\hline
		\textbf{Force} & \textbf{Impact} \\
		\hline
		Threat of New Entrants & The threat of new entrants is high due to low entry barriers in the app development industry. In case a successful counterpart emerges, the application may face serious competition.\\
		\hline
		Bargaining Power of Suppliers & In the context of a barcode scanner, price comparison and shopping assistant application, suppliers could be businesses whose products are listed. They have low bargaining power, as there are numerous online and offline retailers and not being present could lead to missing out on potential customers. \\
		\hline
		Bargaining Power of Buyers & Buyers have moderate bargaining power. As of now, there are only websites available in Hungary for this specific purpose, with more or less relevant features. \\
		\hline
		Threat of Substitute Products or Services & The threat is moderate as there are other ways consumers can seek shopping assistance, such as through the mentioned websites, direct e-commerce platforms, but currently there are no Hungary specific applications on the market. \\
		\hline
		Rivalry Among Existing Competitors & The competition is fierce in the digital marketplace industry, the application needs to provide unique value proposal to be potentially considered, instead of other existing solutions. \\
		\hline
	\end{tabularx}
	\caption{Porter's Five Forces Analysis}
	\label{tab:pffa}
\end{table}

\pagebreak

\subsubsection{PEST Analysis}

The external macro-environment in which "BargainScan" is set out to operate in, can be explored by using the PEST analysis, cosidering political, economic, sociocultural and technological factors. The evaluation of these factors are visible in the table (\ref{tab:pest}) below.

\begin{table}[h]
	\centering
	\begin{tabularx}{\textwidth}{|p{2.3cm}|X|}
		\hline
		\textbf{Factor} & \textbf{Impact} \\
		\hline
		Political & In Hungary various legal regulations affect data privacy, e-commerce, and internet advertising. Any changes in these policies may impact the operation and strategies of the application. \\
		\hline
		Economic & Economic factors such as income, unemployment rates, consumer confidence influence the way and frequency people utilize this application, the willingness of businesses to advertise their products through this platform will determine the success of the application. \\
		\hline
		Sociocultural & The growth of digitalization in every field of everyday life definitely helps the success of the application, since there is a wider spectrum of people who are ready to use their smart devices to make their lives easier. Additionally more and more people aspire to be more conscious about their shopping habits.\\
		\hline
		Technological & Rapid advancements in technology, such as artificial intelligence, machine learning, and the increasing size of mobile network coverage, can lead to more innovative and valuable features in the application. Cybersecurity on the other hand poses a significant challenge for designing such applications. \\
		\hline
	\end{tabularx}
	\caption{PEST Analysis}
	\label{tab:pest}
\end{table}

\subsubsection{Ansoff Matrix Analysis}

A great tool for further clarification of the potential success of the application is the Ansoff matrix. The table (\ref{tab:ansoff}) on the next page will study the strategies such as market penetration, market development, product development and diversification.

\begin{table}[h]
	\centering
	\begin{tabularx}{\textwidth}{|p{3.7cm}|X|}
		\hline
		\textbf{Strategy} & \textbf{Description} \\
		\hline
		Market Penetration & The application will focus primarily on the demographic of daily grocery shoppers, so it will seek to deepen its influence in this existing market. In order to achieve this, the solution will offer unique features, ensure the best deals and unbiased product reviews. \\
		\hline
		Market Development & The app will be designed to be appealing to not only daily shoppers, but also those, who seek to spend their money consciously, by making informed decisions. Moreover, upon a successful adaptation in Hungary, an expansion to other geographic regions is considered part of the long-term plan. \\
		\hline
		Product Development & The fast paced advancement of technology draws a path for continuous technological developments, such as implementing artificial intelligence, machine learning, etc.\\
		\hline
		Diversification & Looking towards the future, the application's idea holds possibilities for expanding into other, new markets. Possibly the biggest potential of these opportunities is held by evolving into an application, that much like Wolt or Foodora, would offer delivering the products from an own warehouse or from existing stores. \\
		\hline
	\end{tabularx}
	\caption{Ansoff Matrix Analysis}
	\label{tab:ansoff}
\end{table}


\pagebreak

\subsection{Stakeholder and User Descriptions}

\subsubsection{Stakeholder Summary}

The table (\ref{tab:ss}) below provides a summary for outlining the key stakeholders involved, highlighting their representation and roles in this mobile application project.

\begin{table}[H]
	\centering
	\begin{tabularx}{\textwidth}{|p{2cm}|X|X|}
		\hline
		\textbf{Name} & \textbf{Represents} & \textbf{Role}\\
		\hline
	Users & Individuals seeking a comprehensive shopping assistant & Provide feedback, requirements, test application \\ \hline
		Retailers & Businesses offering product information for customers & Provide feedback, requirements, test application, contribute to product and store data collection \\ \hline
		Developer & Sole technical expert & Design, develop, test, maintain software solution \\ \hline
	\end{tabularx}
	\caption{Stakeholder Summary}
	\label{tab:ss}
\end{table}

\subsubsection{User Summary}\label{ussect}

The table (\ref{tab:us}) below provides a summary for defining the potential users involved, detailing the way they might interact with the application and their representative stakeholders.

\begin{table}[H]
	\centering
	\begin{tabularx}{\textwidth}{|p{2cm}|X|X|}
		\hline
		\textbf{Name} & \textbf{Description} & \textbf{Stakeholder}\\
		\hline
		General Users & Use the application to scan barcodes, search for products, create shopping lists, provide reviews on items and stores, filter stores based on personal criteria & Users \\ \hline
		Researchers & Use the features of the software for market research purposes & Users \\ \hline
		Admins & Manage the application data regarding stores, products and users & Users \\ \hline
		Retailers & Provide data on their business and goods & Retailers \\ \hline
	\end{tabularx}
	\caption{User Summary}
	\label{tab:us}
\end{table}

\subsubsection{User Environment}

Users are primarily individuals seeking an assistant that helps them make informed decisions about the products they buy and about the businesses they shop at. As mentioned in the user summary (\ref{ussect}), the other types of users are researchers doing market research and retailers, who provide information on their business and products. The goal is to provide a user-friendly mobile application, that implements versatile functionality to fulfill all the important use-cases. The general users are expected to use the application both on the go and at home, while other users will probably use it from an office background.

\pagebreak

\subsection{Product Overview} \label{po}

\subsubsection{Product Perspective}

"BargainScan" is a comprehensive shopping assistant application, developed to help making informed decisions for customers across various demographics. Its positioned within the e-commerce market, targeting users with smartphone devices aged between 12-99 years. The application represents a solution for the overload of redundant information in the retail market, leading to customers being able to make conscious shopping decisions, based on their personal preferences and other consumers experience.

\subsubsection{Summary of Capabilities}

The summary of what functionalities the upcoming application will be able to provide is listed in the table (\ref{tab:cap}) below.
 
\begin{table}[ht]
	\centering
	\begin{tabularx}{\textwidth}{|p{2cm}|X|}
		\hline
		\textbf{Capability} & \textbf{Description} \\ 
		\hline
		Barcode Scanning & Through its barcode scanning abilities, it can effectively look up details on the scanned products. \\ 
		\hline
		Price Comparison & The app ensures users get the most value for their money, comparing prices from multiple retailers and stores in real-time. \\ 
		\hline
		Deal Alerts & Users are notified about price drops for items in their shopping list.\\ 
		\hline
		Reviews & The app offers access to reviews and ratings on products and shops from purchasers, assisting users in making informed decisions. \\ 
		\hline
		Shopping List Management & Users can effortlessly create and manage multiple shopping lists, making planning and buying easier. \\ 
		\hline
		Filter Stores & The app is designed to let users filter the recommended shops by factors such as rating, distance, etc. \\ 
		\hline
		Upload Store and Product Information & The verified retailers are able to upload information about themselves and their products. \\ 
		\hline
	\end{tabularx}
	\caption{Summary of Capabilities}
	\label{tab:cap}
\end{table}

\pagebreak

\subsubsection{Assumptions and Dependencies}

This subsection will discuss a few assumptions and dependencies, which were in mind when coming up with the outline for this application. 

\textbf{Assumptions:}

\begin{itemize}
	\item There is an increasing number of money conscious consumers.
	\item Users can understand and utilize user-friendly applications.
	\item Retailers are interested in using the platform for sharing their prices and deals.
\end{itemize}

\textbf{Dependencies:}

\begin{itemize}
	\item The app is dependent on internet access to provide real-time information.
	\item The functionality could be affected by changes in the operating system of the device.
	\item The effectiveness of the price comparison heavily relies on having information provided by both partner retailers and web scraping.
\end{itemize}

\subsubsection{SWOT Analysis}

To summarize the information disclosed in the product overview (\ref{po}) section, a SWOT analysis will be utilized to evaluate the strengths, weaknesses, opportunities and threats. This evaluation can be seen in the table (\ref{swot}) below.

\begin{table}[ht]
	\centering
	\begin{tabularx}{\textwidth}{|l|X|}
		\hline
		\textbf{Category} & \textbf{Factors} \\ 
		\hline
		Strengths & User-friendly design, covers the whole process of searching, choosing, comparing, filtering products and stores, offers unique features like filtering shops, utilizing web scraping and partnerships with retailers for searching purposes. \\ 
		\hline
		Weaknesses & Reliance on web scraping for product and pricing data and competition from either other solutions or integrated shopping assistants in established e-commerce platforms. \\ 
		\hline
		Opportunities & Expanding smartphone usage, willingness to let smart devices help our daily life and fast-paced technological advancements in AI and Machine Learning for future development possibilities. \\ 
		\hline
		Threats & Fierce competition in the market, rapid technological changes demanding constant product updates, and potential cybersecurity threats. \\ 
		\hline
	\end{tabularx}
	\caption{SWOT Analysis}
	\label{swot}
\end{table}

\pagebreak

\subsection{Product Features}

The application will come with a broad range of features to ensure the best possible experience when it comes to convenience and clarity. The features walk the user through the process of choosing a store that is the most applicable to the users needs, prioritizing price to value. Each feature will be designed with the main goal being ease of use.

As the development progresses and the use case models will be clear, these descriptions will be updated and discussed in further details. The following table (\ref{tab:features}) lists the key features of the upcoming application, detailed on a level that is generally graspable.

\begin{table}[ht]
	\centering
	\begin{tabularx}{\textwidth}{|p{2.5cm}|X|}
		\hline
		\textbf{Capability} & \textbf{Description} \\ 
		\hline
		User Management & Distincts user types and based on the current level provides certain functionality. \\ 
		\hline
		Barcode Scanner & Turns the image of the barcode into a code sequence. \\ 
		\hline
		Product Search & Searches for the product either by name or by code. \\ 
		\hline
		Price Comparison & Creates a sortable list of the shops that are selling the product with price information. \\ 
		\hline
		Deal Notifications & Notifies users if items from their shopping cart are on sale.\\ 
		\hline
		Reviews and Ratings & Lets users write and read ratings for different products and stores. \\ 
		\hline
		Shopping List Management & Lets users create and manage multiple shopping lists. \\ 
		\hline
		Store Filtering & Lets users sort stores based on factors like distance from current position \\ 
		\hline
		Upload Store and Product Information & Lets the retailers provide documents with their pricing data. \\ 
		\hline
	\end{tabularx}
	\caption{Product Features}
	\label{tab:features}
\end{table}

\pagebreak

\subsection{Constraints}

In the context of a shopping assistant mobile application, various constraints could impact the design, development, and implementation.

\begin{itemize}
	\item \textbf{Technological Constraints:} Mobile applications need to be constantly updated to ensure compatibility with the latest versions of smartphone operating systems. Ensuring backwards compatibility might prove to be challenging and resource-intensive.
	\item \textbf{Resource Constraints:} Since it is a thesis work project, limited human resources can slow development down, and the number of features that can be implemented will be reduced within the given timeframe.
	\item \textbf{Security and Privacy Constraints:} Since the application will provide possibility for creating different user profiles, sensitive data will have to be stored, therefore the application will need to comply with strict data privacy laws such as GDPR.
	\item \textbf{Regulatory Constraints:} Depending on the country, the application might need to adhere to various e-commerce and data collection regulations.
	\item \textbf{Time Constraints:} Deadlines for project milestones, such as testing phases and project hand in deadlines, might affect the state of the development and the number of features developed, leading to a change of scope.
	\item \textbf{Market Constraints:} The competitiveness of the market can lead to other applications emerging from nothing, or existing ones expanding with important and attractive features.
	\item \textbf{Integration Constraints:} Since the app relies on third-party services (APIs and web scraping), any changes or issues with these can result in unreliable or broken service.
\end{itemize}

\subsection{Quality Ranges}

This section will describe the target ranges for various quality aspects crucial to the success of "BargainScan". The characteristics include performance, robustness, fault tolerance and usability.

\begin{itemize}
	\item \textbf{Performance:} The application should open, close and react to user interactions swiftly, handle high traffic, otherwise it will lead to a negative experience.
	\item \textbf{Robustness:} The software should be capable of operating smoothly regardless of the conditions. These can be network disruptions or lower system resources.
	\item \textbf{Fault Tolerance:} The app should be able to tolerate exceptions and errors occurring during operation. These problems could arise from external services, the host device or the software itself.
	\item \textbf{Usability:} The program should be easy-to-use for all individuals, providing a clear and straightforward interface.
	\item \textbf{Security:} Even though sensitive payment and financial details are not being utilized in the application, a safe way of authentication and data storing is of key importance.
	\item \textbf{Compatibility:} The application should work on a series of devices from the newest flagships, to at least 5 years old devices, to ensure availability for a wide target audience.
\end{itemize}


\subsection{Documentation Requirements}

This section will describe what documents will be developed along the software development to support a successful application deployment.

\begin{itemize}
	\item \textbf{Release notes:} Release notes will be provided for all versions and updates of the software, describing what is new and what has changed.
	\item \textbf{Integrated help:} There will be an integrated, up-to-date help page in the application, covering the important use cases of the program.
	\item \textbf{Online help:} There will be a dedicated description area discussing the main functionality of the application on the product page of the respective smartphone application store.
\end{itemize}
