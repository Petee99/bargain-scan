The project's scope, which was originally intended as a mobile application, has grown significantly due to web scraping, data collection, administration, and other features. This evolution necessitated the creation of a complete software solution that included an API, database, web application, and the mobile application. 

Given the expanded scope, it was necessary to use a Minimum Viable Product (MVP) approach. This strategic decision was motivated by the realization that developing all of the planned functionality would take longer than expected, and the solution would become obsolete before it was released. Thus, focusing on the MVP allowed me to prioritize core features that would provide immediate value while still leaving room for future development.

\noindent\textbf{Utilization of Current Technologies and Frameworks}

Despite its focus on MVP, the project does not make any compromises in terms of technology. Through the utilization of well-known and cutting-edge technologies and frameworks, such as Angular, MongoDB, ASP.NET and MAUI, the project guarantees that the product will continue to be robust, scalable, and in accordance with the most recent industry standards. These technologies offer a solid basis for future expansions and ensure that the product can adapt to changing user requirements and emerging trends, thus providing a reliable foundation for future expansions.

\noindent\textbf{Modular Design and Clean Coding Principles}

The modular architecture and use of clean coding standards were crucial for dealing with the complexity within a short time span. The modular structure enables for the isolation and development of functions individually, making the system easier to comprehend, test, and develop. Clean coding techniques guarantee that code is maintainable and self-explanatory, which is especially useful when projects are running on tight deadlines and future scalability is a worry. \cite{cleancode}

\noindent\textbf{Continuing Development}

The Agile methodology, with its adaptability and responsiveness, is ideal for the ongoing development of this comprehensive software solution. As the project progresses, agile practices will become increasingly important because they are incremental and iterative. This approach allows for continuous evaluation and integration of feedback, ensuring that all software components, including the API, database, web application, and mobile application, evolve in sync with user needs and technological advancements. \cite{agile}

\pagebreak

By implementing Agile, the project team can maintain flexibility and adjust the development path based on real-world testing and user feedback. This method is especially useful for dealing with the complexity and uncertainty of a multifaceted system design because it allows for frequent re-evaluation and refinement. As a result, Agile reduces the risk of deviating from user expectations or pursuing less impactful features, allowing each release to be as relevant and effective as possible. \cite{agile}

\noindent\textbf{Key Challenges and Future Directions}

One of the key issues faced throughout development was a scarcity of free barcode repositories, which caused a significant bottleneck. This constraint required manual barcode dispensing using cellphones, which proved cumbersome and inefficient. Addressing this issue before final release is important to progressing effectively. Identifying or inventing a simpler, more automated barcode system will be critical for improving user comfort and functionality.

\subsection{Future Prospects and Continuous Improvement}

Looking ahead, the MVP serves as an essential stage toward a more comprehensive system. The architecture's inherent flexibility and clean, modular design allow for the addition of new features and enhancements as time and user feedback permit. The initial ideas for future improvement are technology and infrastructure, user interface and accessibility, and strategic partnerships.

\noindent\textbf{Technology and Infrastructure}

\begin{itemize}
	\item \textbf{Local Database Integration:} Transitioning from storing local-data in a file-based manner to using a local database in the mobile application will be explored to improve the speed and reliability of data retrieval and storage.
	\item \textbf{Web Scraping Enhancements:} It is intended to employ Google's upcoming AI capabilities to enhance and simplify web scraping procedures. The purpose of this improvement is to enable more effective data extraction from online sources, resulting in increased responsiveness and accuracy.
	\item \textbf{iOS Expansion:} It is planned to enter the iOS market in order to broaden the scope of the audience that may be reached.
	\item \textbf{Pricing and Location-Based Algorithms:} Future objectives include developing a price-based calculation algorithm to assist users in selecting the most cost-effective shopping alternatives as well as implementing map-based shopping recommendations based on their location and interests.
\end{itemize}

\pagebreak

\noindent\textbf{User Interface and Accessibility}

\begin{itemize}
	\item \textbf{Online User Profile and Authentication:} Implementing an online user profile and authentication system on the mobile app protects user data, personalizes the experience, and allows for cross-device switching. The API already supports this feature.
	\item \textbf{Two-Tiered Application Model:} There are plans to provide both an ad-supported free version and a premium, ad-free edition, which may contain additional exclusive features for improved functionality and user experience.
\end{itemize}

\noindent\textbf{Strategic Partnerships}

These are crucial for tackling the most important issue regarding the lack of available free barcode repositories the application is facing.

\begin{itemize}
	\item \textbf{Partnerships with Retailers:} Plans include creating strategic alliances with businesses to gain access to their inventory of items and barcodes. This partnership makes it easier to add and accurately identify items inside the application, as well as improve the user experience by offering detailed product data.
	\item \textbf{Partnership with Barcode Repositories:} Collaboration with dedicated barcode storage facilities is also on the plan. This provides access to a diverse set of product barcodes, ensuring full coverage and up-to-date information across several product categories.
\end{itemize}

\noindent\textbf{Ending Note}

This project started as a fundamental mobile app and rapidly grew into a full-fledged software solution that incorporates web scraping, authentication, data management, and user interfaces for both web and mobile platforms. Because of the extensive scope and the need for rapid demonstrable results, MVP strategy was required. This strategy guaranteed that I only had to concentrate on fundamental functions, reducing the danger of outmoded technology before it even reached the market.

One important issue was a scarcity of free barcode containers, which necessitated tedious manual procedures. Addressing issue before the final release will be critical to improve the user experience and operational efficiency.

The MVP's foundations, coupled with its modular design and clean coding techniques, establish the platform for future advancements, allowing the project to adapt and expand in response to user input and developing demands.

